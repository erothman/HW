\documentclass[letterpaper, 11pt]{article}
\usepackage{latexsym}
\usepackage{amssymb}
\usepackage{times}
\usepackage{listings}
\usepackage {tikz}
\usetikzlibrary {positioning}
\usepackage{amsmath,amsfonts,amsthm}
\usepackage{graphicx}
\usepackage[pdf]{graphviz}

\newtheorem{lemma}{Lemma}[section]
\newtheorem{figur}[lemma]{Figure}
\newtheorem{theorem}[lemma]{Theorem}
\newtheorem{proposition}[lemma]{Proposition}
\newtheorem{definition}[lemma]{Definition}
\newtheorem{corollary}[lemma]{Corollary}
\newtheorem{example}[lemma]{Example}
\newtheorem{exercise}[lemma]{Exercise}
\newtheorem{remark}[lemma]{Remark}
\newtheorem{fig}[lemma]{Figure}
\newtheorem{tab}[lemma]{Table}
\newtheorem{fact}[lemma]{Fact}
\newtheorem{test}{Lemma}
\newtheorem{algorithm}[lemma]{Algorithm}

\begin{document}



\title{Homework \#2 \\ Automata and Computation Theory \\Fall 2018}
\author{Written by Eric Rothman}

\maketitle

\section{Problem 1}
Give the state diagram of a finite automaton recognizing the following language. The
alphabet is \{0, 1\}.

\{w $\mid$ w has length exactly 3 and its last symbol is different from its first symbol\}\\
Answer:

\digraph{prob1}{
	rankdir=LR;
	size="4"
	node [shape = doublecircle]; q_1a0 q_0a1;
	node [shape = circle];
	start -> q_1 [ label = "1" ];
	start -> q_0 [ label = "0" ];
	q_1 -> q_1a [ label = "0,1" ];
	q_0 -> q_0a [ label = "0,1" ];
	q_1a -> q_1a0 [ label = "0" ];
	q_1a -> q_fail [ label = "1" ];
	q_0a -> q_0a1 [ label = "1" ];
	q_0a -> q_fail [ label = "0" ];
	q_0a1 -> q_fail [ label = "0,1" ];
	q_1a0 -> q_fail [ label = "0,1" ];
	q_fail -> q_fail [ label = "0,1" ];
}


\newpage
\section{Problem 2}
Give a finite automaton (both a state diagram and a formal description) recognizing the
following language. The alphabet is \{0, 1\}.

\{w $\mid$ w is not the empty string and every odd position of w is a 1\}\\\\
Answer: $M = $

\digraph{prob2}{
	rankdir=LR;
	size="4"
	node [shape = doublecircle]; q_odd q_even;
	node [shape = circle];
	start -> q_odd [ label = "1" ];
	start -> q_fail [ label = "0" ];
	q_odd -> q_even [ label = "0,1" ];
	q_even -> q_odd [ label = "1" ];
	q_even -> q_fail [ label = "0" ];
	q_fail -> q_fail [ label = "0,1" ];
}\\\\
Formal Definiton:\\

$M = \{\{ start, q\_odd, q\_even, q\_fail \}, \{0, 1\}, \delta, start ,\{q\_even, q\_odd \}\}$\\\\

$\delta = $
\begin{tabular}{|c|c|c|}
  \hline
   & 0 & 1\\
  \hline
  start & q\_fail & q\_odd\\
  \hline
  q\_odd & q\_even & q\_even\\
  \hline
  q\_even & q\_fail & q\_odd\\
  \hline
  q\_fail & q\_fail & q\_fail\\
  \hline
\end{tabular}
\newpage
\section{Problem 3}
Show that the following language is regular, where the alphabet is \{0, 1\}.

\{w $\mid$ w contains an equal number of occurrences of the substrings 01 and 10\}\\\\
Answer:

A simpler way to write out the language is to observe that everytime the string returns to its starting character, there are an equal number of substrings of 01 and 10.
I will prove why this is in the proof section of the answer.
For a language to be regular, it needs to be the language for a finite automata.
I claim and will prove that the given language, $A$, is the equal to the language, L(M), for the following automata:

\digraph{prob3}{
	rankdir=LR;
	size="4"
	node [shape = doublecircle]; start q_0 q_1;
	node [shape = circle];
	start -> q_0 [ label = "0" ];
	start -> q_1 [ label = "1" ];
	q_0 -> q_01 [ label = "1" ];
	q_1 -> q_10 [ label = "0" ];
	q_01 -> q_0 [ label = "0" ];
	q_10 -> q_1 [ label = "1" ];
	q_0 -> q_0 [ label = "0" ];
	q_1 -> q_1 [ label = "1" ];
	q_01 -> q_01 [ label = "1" ];
	q_10 -> q_10 [ label = "0" ];
}

\subsection*{Claim 1}
I will prove \{w $\mid$ w contains an equal number of occurrences of the substrings 01 and 10\} = \{w $\mid$ w starts and ends on the same symbol\}

\subsubsection*{First Direction}
Prove \{w $\mid$ w contains an equal number of occurrences of the substrings 01 and 10\} $\subset$ \{w $\mid$ w starts and ends on the same symbol\}

let $a \in A$,

So a has an equal number of occurrences of the substrings 01 and 10

WLOG let a start with 0

Every time the substring 01 appears in a, the current character is a 1.

Since there are an equal number of 01 and 10 substrings in a and since to hava 01 substring, the current symbol must be a 0, the next occurance of one of those two substrings must be 10.

So every time 01 appears, a 10 must appear before another version of 01.

Since there are an equal number of substrings, every 01 substring is followed by a 10 substring. Otherwise there couldn't be an equal amount since either substring needs the other one between iterations.

So the last substring to appear is a 10 substring.

So a ends in a 0 and starts in a 0

So a $\in$ \{w $\mid$ w starts and ends on the same symbol\}

So \{w $\mid$ w contains an equal number of occurrences of the substrings 01 and 10\} $\subset$ \{w $\mid$ w starts and ends on the same symbol\}
\subsubsection*{Other Direction}
Prove \{w $\mid$ w starts and ends on the same symbol\} $\subset$ \{w $\mid$ w contains an equal number of occurrences of the substrings 01 and 10\}

Let a $\in$ \{w $\mid$ w starts and ends on the same symbol\}

So a starts and ends with the last symbol.

WLOG let a start and end in 0

Either there are no 1 in a, in which case there are 0 appearances of both 10 and 01 so a $\in$ A, or there is atleast a 1 in a

Since a starts and ends on the same symbol, for every appearance of the substring 01, when the current charactor becomes a 1 from a 0, there must eventually be an occurance of 10, otherwise a couldn't end in 0.

Since for every appearance of 01 there is a 10 associated with it, there are an equal number of 01 and 10 substrings in a

So a $\in$ A

So \{w $\mid$ w starts and ends on the same symbol\} $\subset$ \{w $\mid$ w contains an equal number of occurrences of the substrings 01 and 10\}

Since both directions hold up, \{w $\mid$ w starts and ends on the same symbol\} = \{w $\mid$ w contains an equal number of occurrences of the substrings 01 and 10\}
\subsection*{Proof}
Prove that L(M) = A.

Let B = \{w $\mid$ w starts and ends on the same symbol\}
\subsubsection*{First Direction}
Prove that L(M) $\subset$ A

FSOC let L(M) $\not\subset$ A

So $\exists b \in L(M)$ such that $b \notin A$

Since $b \notin A$, $b \notin B$ by Claim 1

Since $b \notin B$, b must start and end with different symbols.

WLOG let b start with 0 and end with 1

Since b starts with 0, after the 0 is passed, the current state would be q\_0.

Since the current state is q\_0, everytime a 1 is encountered the state becomes q\_01.

Since b ends with 1, the last state is q\_01.

This contradicts the fact that q\_01 is not an accept state but b $\in$ L(M) from the assumption.

Since there is a contradiction when assumed false, L(M) $\subset$ A.

Since there is a finite automata M for which A is the recognized language, A is a regular language.

\subsubsection*{Other direction}
Prove A $\subset$ L(M)

FOSC let A $\not\subset$ L(M)

So $\exists b \in A$ such that $b \notin L(M)$

Since b $\notin$ L(M), after it is fed through M, the ending state is either q\_01 or q\_10.

WLOG let the ending state be q\_01

Since the ending state is q\_01, the first symbol in b must be 0 and the last one must be 1.

Since the first symbol in b is 0 and the last one is 1, the first and last symbols of b are not equal.

So $b \notin B$.

This contradicts Claim 1 since $b \in A$ but $b \notin B$ and by Claim 1, A = B

Since there is a contradiction when assumed false, $A \subset L(M)$\\

Since A $\subset$ L(M) and L(M) $\subset$ A, L(M) = A.
\newpage
\section{Problem 4}
For any string $w = w_1w_2 · · · w_n$, the reverse of w, written as $w^R$, is the string w in reverse
order, $w_n$ · · · $w_2w_1$. For any language A, let $A^R$ = $\{w^R|w \in A\}$. Show that if A is regular, so is $A^R$\\\\
Proof:

\subsection*{Claim 1}
If A is regular and it has only 1 accept state, then $A^R$ is regular.

PROOF by construction:
Since A is regular, it has a FA, $M = \{Q_1, \mathcal{E}, \delta, q_1, F_1 \}$, that recognizes it, so $L(M) = A$

Since there is only one accept state, $|F_1| = 1$. Let $f_1$ be the only value in $F_1$

To construct a FA that recognizes $A^R$, we modify $M$ into a NFA $M^R = \{Q_1, \mathcal{E}, \delta', f_1, \{q_1\}\}$ where $\delta'$ contains the flipped version of every transition.
So if there was a transition from $q_a$ to $q_b$ in $\delta$, there is a transition with the same requirement fom $q_b$ to $q_a$ in $\delta'$.\\
Now we just need to show that $L(M^R) = A^R$.

Let the sequence of states $M$ visits for input $w$ be $p_1, p_2, ..., p_{n+1}$

$w^R \in A^R <==>$ 

$w \in A <==> $

$w \in L(M^R) <==> $

$p_{n+1} \in F_1 and p_1 = q_1 <==>$

$p_{n+1} = f_1 <==>$

$p_{n+1} ... p_2, p_1$ is a valid sequence in $M^R <==>$

$w \in L(M^R)$

So $A^R = L(M^R)$.

Since an NFA recognizes the language, $A^R$ is a regular language.

\subsection*{General Proof}
I will prove that if A is regular, then $A^R$ is regular.

I will prove so using induction on the number of accepted states in the finite automata $M$ that recognizes A.

\subsubsection*{Base Case}
n = 1, this was proven in claim 1
\subsubsection*{Induction Step}
IH: $\exists k \in \mathcal{N}$ such that for all recognized languages A with a finite automata M that recognizes it and has $k$ accepted states, $A^R$ is a recognized language.

Lets look at when M that has $k + 1$ accepted states.

let $F = \{a_1, a_2 ... , a_{n+1}\}$ where $a_i$ is the ith accepted state.

So $M = \{Q, \mathcal{E}, \delta, q_0, F \}$

Lets divide this into two languages C and D where C is the language recognized by $M_1 = \{Q, \mathcal{E}, \delta, q_0, F \ \{a_{n+1}\}\}$ and D is the language recognized by $M_2 = \{Q, \mathcal{E}, \delta, q_0, \{a_{N+1}\}\}$.

So $C \cup D = A$ because D contains all strings that end on $a_{n+1}$ when put through $M$, and C contains all strings that end in anything besides $a_{n+1}$ that is accepted.
The actual proof is shown below.\\

Let $a \in C \cup D$, so $a \in C$ or $a \in D$.

If $a \in C$ then a, when run through M, ends on an accept state, So $a \in A$

If $a \in D$ then a, when un through M, ends on $a_{n+1}$ which is an accept state, so $a \in A$\\

Now let $b \in A$, so when run through M, the last state is a state in the set of accept states.

If the last state, when b was run through M is $a_{n+1}$, then $b \in D$ so $b \in C \cup D$

If the last state, when b was run through M is not $a_{n+1}$, then $b \in C$ since $b$ had to land on an accept state, so $b \in C \cup D$ \\

Since $C \cup D = A$ , $C^R \cup D^R = A^R$.
This is proven below:\\

Let $a^R \in C^R \cup D^R$ be an arbitrary element.

so $a^R \in C^R$ or $a \in D^R$

If $a^R \in C^R$, $a \in C$ 
so $a \in C \cup D$ 
so $a \in A$ since $C \cup D = A$
so $a^R \in A^R$

If $a^R \in D^R$, $a \in D$ 
so $a \in C \cup D$ 
so $a \in A$  since $C \cup D = A$
so $a^R \in A^R$

so either way, $a^R \in A^R$, so $C^R \cup D^R \subset A^R$\\

Now let $a^R \in A^R$ be an arbitrary element.

Since $a^R \in A^R$, $a \in A$

So $a \in C \cup D$ since $C \cup D = A$

So $a \in C$ or $a \in D$

If $a \in C$ then $a \in C^R$
so $a \in C^R \cup D^R$

If $a \in D$ then $a \in D^R$
so $a \in C^R \cup D^R$

so $A^R \subset C^R \cup D^R$\\

Since $A^R \subset C^R \cup D^R$ and $C^R \cup D^R \subset A^R$, $C^R \cup D^R = A^R$\\

Since C and D are defined of be the languages recognized by automata, they are regular languages.

Since C is a regular language and the number of accept states in $M_1$ is $n$, by IH $C^R$ is a regular language.

Since D is a regular language and the number of accept states in $M_2$ is $1$, by claim 1 $D^R$ is a regular language.

Since $C^R$ and $D^R$ are regular languages, $C^R \cup D^R = A^R$ and regular languages are closed under union, $A^R$ is a regular language.

So by induction for every regular language $A$ whose finite automata has $n >= 1$ accept states, $A^R$ is a regular language.
\end{document}